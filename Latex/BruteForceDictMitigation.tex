\documentclass{comjnl}

\usepackage[utf8]{inputenc}       
\usepackage[slovene]{babel}
\usepackage{booktabs}
\usepackage[table,xcdraw]{xcolor}
\usepackage{caption}
\usepackage{hyperref}


\captionsetup{
  singlelinecheck = false, format= hang, justification=raggedright
}

%% These two lines are needed to get the correct paper size
%% in TeX Live 2016
\let\pdfpageheight\paperheight
\let\pdfpagewidth\paperwidth

%\copyrightyear{2018} \vol{00} \issue{0} \DOI{000}

\begin{document}
\author{João Bettencourt}
\title{Brute-force, Dictionary Attacks and Mitigation}
\affiliation{Fakulteta za elektrotehniko, računalništvo in informatiko, Univerza v Mariboru} \email{joao.bettencourt@um.si}

\shortauthors{J. Bettencourt}
 

\keywords{Brute-Force Attacks, Dictionary Attacks, Password Security, Cybersecurity.}


\begin{abstract}
The increasinng volume on digital systems in the modern society has increased the importance of security measures.
One of the most persistent threats to digital security is the brute-force and dictionary attack, which exploits weak passwords to be harmful to systems. 
This paper explores these attack techniques, their mechanisms, and their impact on the modern society. 
By understanding the key points behind brute-force and dictionary attacks, we can identify vulnerabilities in password systems and propose effective mitigation strategies. 
This topic was chosen due to its significant relevance in today's digital age, where establishing security standards and practices is essential for protecting sensitive information.
The study also emphasizes the need for stronger authentication methods and awareness to protect against evolving attack techniques. 
By combining theoretical analysis with practical demonstrations using tools such as Kali Linux, this work aims to highlight not only the risks posed by these attacks but also measures to improve digital security.
\end{abstract}

\maketitle


\section{Introduction}
\label{Sec:Sklicnapoglavje}
Password-based authentication remains one of the most common methods for system security, based on their simplicity and effiency. However, it continues to show significant vulnerabilities, facing brute-force and dictionary attacks. Those methods exploit the inherent weaknesses of user-created passwords, which are often short, predictable and based on common patterns. In spite of decades of research and development of numerous countermeasures, the human factor remains the weakest link in the security trifecta.

By the new advancements in computational power, these vulnerabilities got aggravated, making attackers more effient cracking passwords. A task that used to require days now takes seconds or even less. At the same time, passwords cracking techniques have become progressively sophisticated, including vast dictionaries, permutations, and adaptive algorithms. These developments highlight the critical need for powerfull mitigation strategies that not only address technical vulnerabilities but also focus on improving user practices 

This study investigates the mechanics of brute-force and dictionary attacks, demonstrates their effectiveness using practical experiments, and evaluates their strengths and weaknesses of most commonly used mitigation strategies. By analyzing passwords, by their entropy and characteristics, this work aims to emphasize the importance of adopting strategies to mitigate these attacks, by adopting stronger password policies and multi-factor authentication methods.

\section{Execution of the Attacks}

Attacks... 

\subsection{Regular Expressions}

Regular Expressions and What was used in this project

\subsection{Brute-force Approach}

How the Brute-force Approach happened

\subsection{Dictionary attack Approach}

How the Dictionary Approach happened

\section{Analysis}

Analysis of what was done before

\section{Mitigation Strategies}

Short description of what is a mitigation strategie

\subsection{Account Lockouts}
After X number of attempts locks and account

\subsection{Captcha}

Side bard captcha to prevent software to imitate human behavior, such as bots or botnets


\subsection{Multi-factor Authentication}
Multi-factor authentication most known as "MFA" enhances security by requiring useres to provide more than one form of verification, usually 2 to 3 ways of verification, before accessing a system. By combining something the user knows, something they have, or something they are, MFA adds layers of protection that forbid unauthorized access. Research indicates that implementing MFA can block over 99.9\% of account compromising attacks, including brute-force and dictionary attempts.\cite{mfa_effectiveness} Within the frame of brute-force attacks, where attackers widely attempt various password combinations to gain unauthorized acess, MFA serves as a tough brake. Even if an attacker is sucessfull guessing a password, the aditional verification factors required by MFA, prevent unauthorized entries. This multi-layered system ensures that the compromise of a single factor does not grant access, thus significantly enhancing security.[11]\cite{mfa_brute_force} For that reason, adopting MFA is a crucial strategy to mitage risks associated with brute-force attacks, ensuring that only authorized userss can access sensitive systems and informations.

\subsection{Password lenght}
Password lenght is a critical factor to defend against brute-force attacks, where attackers can attempt multiple times all possible to estimate a password. The longer the passwords the longer it takes to be deciphered, increasing the number of potential combinations. Refering to a study analyzing real-world passwords found that shorter passwords are more receptive to brute-force attacks. with over 11\% of them being cracked. \cite{password_lenght} In addition , research indicates that a significant portion of compromised passwords are under 12 characters, suggesting that increasing password lenght can improve resistance to these attacks. \cite{specops} For that reason, enforcing longer password policies is an effective strategy to mitage the risk of digital break-ins. 

\subsection{SALT}
adding salt to when a password is hashed in the database

\section{Conclusion}

\section{References}
\begin{thebibliography}{99}

  \bibitem{password_lenght} Password Cracking with Brute Force Algorithm and Dictionary Attack Using Parallel Programming
  \textit{MDPI}, \url {https://www.mdpi.com/2076-3417/13/10/5979}

  \bibitem{specops} Do longer passwords protect you from compromise?
  \textit{specops}, \url {https://specopssoft.com/blog/longer-passwords-protect-compromise/}

  \bibitem{mfa_effectiveness} How effective is multifactor authentication at deterring cyberattacks? 
  \textit{Cornell University}, \url{https://arxiv.org/abs/2305.00945}.
  
  \bibitem{mfa_brute_force} Multi-Factor Credential Hashing for Asymmetric Brute-Force Attack Resistance. 
  \textit{Cornell University}, \url{https://arxiv.org/abs/2306.08169v1}.
  
\end{thebibliography}


\bibliographystyle{compj}
\bibliography{literatura}


\end{document}